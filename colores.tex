\documentclass[a4paper]{article}
\usepackage[latin1]{inputenc}
\usepackage{color}
\usepackage{url}
\usepackage{todonotes}
\usepackage{soul}

\newcommand{\notajj}[1]{\todo[size=\small,linecolor=green,backgroundcolor=green!90!white,bordercolor=purple]{#1  \\ -- \footnotesize \color{purple} JJ}}
\newcommand{\notapedro}[1]{\todo[size=\small,linecolor=blue,backgroundcolor=blue!20!white,bordercolor=blue]{#1  \\ -- \footnotesize \color{blue} Pedro}}
\newcommand{\notaantonio}[1]{\todo[size=\small,linecolor=red,backgroundcolor=orange!20!white,bordercolor=red]{#1  \\ -- \footnotesize \color{orange} Antonio}}
\newcommand{\notamaribel}[1]{\todo[size=\small,linecolor=red,backgroundcolor=orange!20!white,bordercolor=red]{#1  \\ -- \footnotesize \color{red} Maribel}}
% -| Vrivas - 30-Nov-2014 |-----------------------------------------------------------
\definecolor{cvrivas}{rgb}{0.9,0.4,0.4}
\newcommand{\notavrivas}[1]{\todo[size=\small,linecolor=cvrivas,backgroundcolor=cvrivas!20!white,bordercolor=white]{#1 \\ -- \footnotesize \color{cvrivas} Victor}}
\definecolor{cfergu}{rgb}{0.2,0.5,0.9}
\newcommand{\notafergu}[1]{\todo[size=\small,linecolor=cfergu,backgroundcolor=cfergu!20!white,bordercolor=black]{#1 \\ -- \footnotesize \color{cfergu} Fergu}}
\definecolor{cpaloma}{RGB}{152,251,152}
\definecolor{cbpaloma}{RGB}{0,100,0}
\newcommand{\notapaloma}[1]{\todo[size=\small,linecolor=cbpaloma,backgroundcolor=cpaloma!20!white,bordercolor=cbpaloma]{#1 \\ -- \footnotesize \color{cbpaloma} Paloma}}
\newcommand{\notagus}[1]{\todo[size=\small,backgroundcolor=yellow!33,bordercolor=yellow,linecolor=yellow]{#1  \\ \footnotesize -- Gustavo}}
\newcommand{\notag}[2]{\hl{#1}\todo{#2}}

% ------------------------------------------------------------------------------------
\begin{document}

\title{Eligiendo nuestros colores para \notag{todonotes}{En Fedora el paquete se llama texlive-todonotes... �No queda m�s bonito as�n? Inconveniente: a�adir soul}}

\author{Geneura at large}

\maketitle

% esta directiva es la que incluye justo antes del abstract la lista 
% de comentarios que se hayan incluido en el resto del paper
\listoftodos


\begin{abstract}
 Vamos a elegir cada uno un color para trabajar con {\tt todonotes} \notavrivas{Colorin, colorado, esta nota se ha acabado.} \notamaribel{A m� me salen las notas de todos dobles.} para hacer nuestras anotaciones. 
 Hay que instalarlo de
\url{http://ctan.org/pkg/todonotes} y luego a�adir una anotaci�n aqu� \notapaloma{�Aqu�? Pues aqu� la a�ado.}
o donde sea que indique el color que vamos a elegir. En el manual (y     \notapedro{�Vamos a probarlo!}
ah� m�s arriba en el fuente) explica c�mo hacer una macro para cada   \notafergu{Nota hecha por un dalt�nico}
uno.
En un lugar de la Mancha\notagus{�Seguro que era la Mancha?}...

\end{abstract}

\section{Acknowledgements}\notajj{Aprovechamos y metemos aqu� los proyectos a los que haya que
  agradecerles en cada momento}

Yo creo que mejor ponerlos en una lista, cada uno con su frase:

This work has been supported in part by:
\begin{itemize}
\item SIPESCA under Programa Operativo FEDER de Andaluc�a 2007-2013
\item TIN2011-28627-C04-02 awarded by the Spanish Ministry of Economy and Competitivity
\item SPIP2014-01437 (Proyecto PETRA, funded by Direcci�n General de Tr�fico)
\item Este proyecto con n� de referencia: PRY142/14 ha sido financiado �ntegramente por la Fundaci�n P�blica Andaluza Centro de Estudios Andaluces en la IX Convocatoria de Proyectos de Investigaci�n. \notamaribel{Este proyecto tiene que ponerse esta frase tal y como est�, en espaniol, por exigencias del contrato.}
\item PYR-2014-17 GENIL project, awarded by CEI-BIOTIC Granada. \notaantonio{Borrad los antiguos y poned el m�o, hombre ya! :D}
\end{itemize} 



A m� las notas me destrozan el texto, sale todo descuadrado.


\end{document}

\documentclass[a4paper]{article}
\usepackage[latin1]{inputenc}
\usepackage{color}
\usepackage{url}
\usepackage{todonotes}


\newcommand{\notajj}[1]{\todo[size=\small,linecolor=green,backgroundcolor=green!90!white,bordercolor=purple]{#1  \\ -- \footnotesize \color{purple} JJ}}
\newcommand{\notapedro}[1]{\todo[size=\small,linecolor=blue,backgroundcolor=blue!20!white,bordercolor=blue]{#1  \\ -- \footnotesize \color{blue} Pedro}}
% -| Vrivas - 30-Nov-2014 |-----------------------------------------------------------
\definecolor{cvrivas}{rgb}{0.9,0.4,0.4}
\newcommand{\notavrivas}[1]{\todo[size=\small,linecolor=cvrivas,backgroundcolor=cvrivas!20!white,bordercolor=white]{#1 \\ -- \footnotesize \color{cvrivas} Victor}}
% ------------------------------------------------------------------------------------
\begin{document}

\title{Eligiendo nuestros colores para todonotes}

\author{Geneura at large}

\maketitle

% esta directiva es la que incluye justo antes del abstract la lista 
% de comentarios que se hayan incluido en el resto del paper
\listoftodos


\begin{abstract}
 Vamos a elegir cada uno un color para trabajar con {\tt todonotes} \notavrivas{Colorin, colorado, esta nota se ha acabado.}
para hacer nuestras anotaciones. 
 Hay que instalarlo de
\url{http://ctan.org/pkg/todonotes} y luego a�adir una anotaci�n aqu�
o donde sea que indique el color que vamos a elegir. En el manual (y     \notapedro{�Vamos a probarlo!}
ah� m�s arriba en el fuente) explica c�mo hacer una macro para cada
uno.
\end{abstract}

\section{Acknowledgements}

This work has been supported in part by
FPU research grant AP2009-2942 and projects SIPESCA (under Programa
Operativo FEDER de Andaluc�a 2007-2013) and TIN2011-28627-C04-02
awarded by the Spanish Ministry of Economy and Competitivity.\notajj{Aprovechamos y metemos aqu� los proyectos a los que haya que
  agradecerles en cada momento}

\end{document}
